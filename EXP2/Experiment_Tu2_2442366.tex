\documentclass[letterpaper,12pt]{article}
\usepackage{tabularx} % extra features for tabular environment
\usepackage{amsmath}  % improve math presentation
\usepackage{graphicx} % takes care of graphic including machinery
\usepackage[margin=1in,letterpaper]{geometry} % decreases margins
\usepackage{cite} % takes care of citations
\usepackage[final]{hyperref} % adds hyper links inside the generated pdf file
\hypersetup{
	colorlinks=true,       % false: boxed links; true: colored links
	linkcolor=blue,        % color of internal links
	citecolor=blue,        % color of links to bibliography
	filecolor=magenta,     % color of file links
	urlcolor=blue         
}
%++++++++++++++++++++++++++++++++++++++++


\begin{document}

\title{Experiment 2 \protect\\Signal Generators and Oscilloscopes}
\author{Ahmet Akman \protect\\ Assistant : Etki Açılan}
\date{\today}
\maketitle

%\begin{abstract}
%abstract
%\end{abstract}


\section{Introduction} 
In this experiment, as students we are expected to experiment how to use the function generators and oscilloscopes by completing the steps described in second experiment laboratory manual. Throughtout these steps, the monitoring properties of the oscilloscope , different output forms of the signal generator observed via connecting the instruments directly each other and to the circuit. The results of the steps noted and plotted for the further comments.
\section{Experimental Results}
In this section the results of the Experiment 2 are discussed.
\subsection{Step 1}
In this step only the oscilloscope instrument used. The probes of the "Channel 1" is connected to the "Prob Comp" terminals which supplies square waveform. After adjusting the monitor scale using "AutoScale" button, peak to peak value is observed from the grids of the monitor. Peak to peak voltage is measured as "2.40 Volts". When "Vertical Division Control" rotary button adjusted to different positions the voltage values of which the grids represents differs. When "Vertical Position Control" rotary button adjusted to diferent positions the signal moves in vertical position without changing its from. The "Horizontal Division Control" rotary button is used to scale the time steps for the active signal. The "Horizontal Position Control" is used to regulate the delay of the signal display. By observing the channel axis control buttons' functions the step 1 is completed.
\subsection{Step 2}
\subsection{Step 3}
\subsection{Step 4}
\subsection{Step 5}
\subsection{Step 6}
\subsection{Step 7}
\subsection{Step 8}
\subsection{Step 9}
\subsubsection{a.}
\subsubsection{b.}


\section{Conclusion}
\section{Appendix I}

%++++++++++++++++++++++++++++++++++++++++
% References section will be created automatically 
% with inclusion of "thebibliography" environment
% as it shown below. See text starting with line
% \begin{thebibliography}{99}
% Note: with this approach it is YOUR responsibility to put them in order
% of appearance.

\begin{thebibliography}{99}

%https://tr.overleaf.com/latex/templates/sample-lab-report-for-u-of-r-phys-349/pgsyqngcyjxk

\end{thebibliography}


\end{document}